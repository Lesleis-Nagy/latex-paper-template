\documentclass[11pt]{article}
\usepackage{caption}

\usepackage{pdfsync}
\usepackage[T1]{fontenc}
\usepackage[latin9]{inputenc}
\usepackage[fleqn]{amsmath}
\usepackage{amsfonts}
\usepackage{graphicx}
\usepackage{longtable}
\usepackage{pdflscape}
\usepackage{cancel}
\usepackage{luximono}
\usepackage{lscape}
\usepackage{color}
\usepackage{float}
\usepackage{subfig}
\usepackage{tikz}
\usepackage{hyperref}
\usepackage{url}
\usepackage{listings}
\usepackage{subfig}
\usepackage{xfrac}
\usepackage{stmaryrd}
\usepackage{titling}
\usepackage{lipsum}
\usetikzlibrary{shapes}
\usepackage{mathrsfs}

%%%%%%%%%%%%%%%%%%%%%%%%%%%%%% Caption.
\usepackage[font=small,labelfont=bf,width=1.0\textwidth]{caption}

%%%%%%%%%%%%%%%%%%%%%%%%%%%%%% Geometry.
\usepackage{geometry}
\geometry{verbose,a4paper,tmargin=3cm,bmargin=3cm,lmargin=3cm,rmargin=3cm,headheight=2cm,headsep=2cm,footskip=1cm}

\setlength{\parskip}{\medskipamount}
\setlength{\parindent}{0pt}

%%%%%%%%%%%%%%%%%%%%%%%%%%%%%% Remove date on title.
\usepackage{titling}
\predate{}
\postdate{}

%%%%%%%%%%%%%%%%%%%%%%%%%%%%%% Bibliography
\usepackage{biblatex} %Imports biblatex package
\addbibresource{bibliography.bib} %Import the bibliography file

%%%%%%%%%%%%%%%%%%%%%%%%%%%%%% Theorems, results, definitions, etc.
\usepackage{amsthm}
\theoremstyle{definition}
\newtheorem{axiom}{Axiom}
\newtheorem{theorem}{Theorem}
\newtheorem{result}{Result}
\newtheorem{definition}{Definition}
\newtheorem{exercise}{Q}
\newtheorem{example}{Example}
\newtheorem{eqn}{Equation}

%%%%%%%%%%%%%%%%%%%%%%%%%%%%%% Headers and footers.
%\usepackage{fancyhdr}
%\pagestyle{fancy}
%\cfoot{}
%\rfoot{SID: s1025371\\Exam no: B004188}
%\lfoot{DAPA Exercise set 2}

%%%%%%%%%%%%%%%%%%%%%%%%%%%%%% Macros
\newcommand{\vari}[1]{\textcolor{red}{#1}}
\newcommand{\varj}[1]{\textcolor{blue}{#1}}
\newcommand{\varjnew}[1]{\textcolor{green}{#1}}
\newcommand{\len}{\ensuremath{\text{len}}}
\newcommand{\curl}{\ensuremath{\text{curl}}}
\newcommand{\pdiff}[2]{\ensuremath{\frac{\partial #1}{\partial #2}}}

\newcommand{\grad}{\ensuremath{\nabla}}
\newcommand{\cross}{\ensuremath{\times}}

% inline code macro
\newcommand{\ilcode}[1]{\footnotesize\texttt{#1}\normalsize}

\newcommand*\mycirc[1]{%
  \begin{tikzpicture}
    \node[draw,circle,inner sep=1pt] {#1};
  \end{tikzpicture}}

% Vectors
\newcommand{\vect}[1]{\boldsymbol{#1}}

% Boundary integrals
\newcommand{\oiint}[1]{%
  \tikz \node[draw, ellipse, inner xsep=-3pt, inner ysep=-9pt] {$\displaystyle\iint_{#1}$};%
}

\newcommand*\oiiint{%
  \tikz \node[draw, ellipse, inner xsep=-4pt, inner ysep=-9pt] {$\displaystyle\iiint$};%
}

% Allows us to quote unit within unit brackets
\newcommand{\ubrackets}[1]{\ensuremath{ \left\llbracket #1 \right\rrbracket }}

\hypersetup{
  colorlinks  = true,
  linkcolor   = blue,    % NOTE: change (blue) to (colIdentifier) to have links within the document in Black
  anchorcolor = blue,
  citecolor   = red,     % NOTE: change (blue) to (colIdentifier) to have citation links within the document in Black
  filecolor   = green,
  urlcolor    = blue
}

\definecolor{CommentColor}{cmyk}{0.1, 0.8, 0. 0.1}

\lstset{
  language={[GNU]C++},
  inputencoding=utf8,
  basicstyle=\scriptsize\ttfamily,
  breaklines=true,
  breakatwhitespace=true,
  tabsize=4,
  numbers=left, % no line nb
  numberstyle=\tiny,
  xleftmargin=0.04\textwidth,
  frame=tB, % single frame around the code
  captionpos=b, % caption to bottom
  emph={Q\_OBJECT, MPI\_COMM\_WORLD, NULL}, % adding the MPI macros
  emphstyle=\textit,
  morecomment=[l][\color{CommentColor}]{//},         % // comments in green
  morecomment=[s][\color{CommentColor}\textbf]{/*}{*/}, % /* */ comments in red => important
  aboveskip=20pt,
  belowskip=10pt
}

% An environment for algorithms/pseudocode

\newcounter{nalg}[section] % defines algorithm counter for section-level
\renewcommand{\thenalg}{\arabic{nalg}} %defines appearance of the algorithm counter
\DeclareCaptionLabelFormat{algocaption}{Algorithm \thenalg} % defines a new caption label as Algorithm x.y

\lstnewenvironment{algorithm}[1][] %defines the algorithm listing environment
{   
    \refstepcounter{nalg} %increments algorithm number
    \captionsetup{labelformat=algocaption,labelsep=colon} %defines the caption setup for: it ises label format as the declared caption label above and makes label and caption text to be separated by a ':'
    \lstset{ %this is the stype
        mathescape=true,
        frame=tB,
        numbers=left, 
        numberstyle=\tiny,
        basicstyle=\scriptsize, 
        keywordstyle=\color{black}\bfseries\em,
        keywords={,input, output, return, datatype, function, in, if, else, foreach, while, begin, end, } %add the keywords you want, or load a language as Rubens explains in his comment above.
        numbers=left,
        xleftmargin=.04\textwidth,
        #1 % this is to add specific settings to an usage of this environment (for instnce, the caption and referable label)
    }
}
{}

%%%

\makeatletter
\renewcommand*\env@matrix[1][\arraystretch]{%
  \edef\arraystretch{#1}%
  \hskip -\arraycolsep
  \let\@ifnextchar\new@ifnextchar
  \array{*\c@MaxMatrixCols c}}
\makeatother

\synctex=1

% Proof environment
\newcounter{proofc}
\renewcommand\theproofc{(\arabic{proofc})}
\DeclareRobustCommand\stepproofc{\refstepcounter{proofc}\theproofc}
\newenvironment{twoproof}{\tabular{@{\stepproofc}c|l}}{\endtabular}

% Proof numbering
\numberwithin{equation}{section}

